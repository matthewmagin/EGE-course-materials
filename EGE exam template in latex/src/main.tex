\documentclass[12pt]{article}
\usepackage[left=1cm,right=1cm,
    top=1cm,bottom=1cm,bindingoffset=0cm]{geometry}
\usepackage{tipa}
\usepackage{euscript}	 
\usepackage{mathrsfs} 
%\usepackage{txfonts}
\usepackage{textcomp}
\usepackage{enumitem}
\usepackage{cmap}
\usepackage{multicol}					
\usepackage{mathtext} 				
\usepackage[T2A]{fontenc}			
\usepackage[utf8]{inputenc}			
\usepackage[english,russian]{babel}	
\usepackage{amsfonts}
\usepackage{amsmath,amsfonts,amssymb,amsthm,mathtools} % AMS
\usepackage{icomma} % "Умная" запятая: $0,2$ --- число, $0, 2$ --- перечисление
\usepackage{fancybox,fancyhdr} %this packages provides fancy up and bottom of page
\usepackage{setspace}
%\полуторный интервал

\usepackage{lscape}
\usepackage{arcs}
\usepackage{import}
\usepackage{xifthen}
\usepackage{pdfpages}
\usepackage{transparent}
\usepackage{caption}
\usepackage{graphicx}
\graphicspath{{pictures/}}
\DeclareGraphicsExtensions{.pdf,.png,.jpg}
\newcommand{\incfig}[1]{%
    \def\svgscale{1.32}
    \import{./figures/}{#1.pdf_tex}
}
\begin{document}
\pagestyle{empty}
\begin{landscape}
\begin{center}
    \section*{Пробный вриант  ЕГЭ по профильной математике.}
    \subsection*{Месяц год, вариант №$N$.}
\end{center}
\begin{multicols}{2}
    \textbf{\textit{Разбалловка задач:}}
    \begin{enumerate}[start=1,label={\itshape\bfseries \arabic*.}]
        \item 1 балл
        \item 1 балл
        \item 1 балл
        \item 1 балл
        \item 1 балл
        \item 1 балл
        \item 1 балл
        \item 1 балл
        \item 1 балл
        \item 1 балл
        \item 1 балл
        \item а) 1 балл\\
                б) 1 балл
        \item а) 1 балл\\
            б) 2 балла
        \item 2 балла
        \item 2 балла
        \item а) 1 балл\\
            б) 2 балла
        \item 4 балла
        \item а) 1 балл\\
            б) 1 балл\\
            в) 2 балла
    \end{enumerate}
    \columnbreak

    \textbf{\textit{Результаты:}}\\

    \begin{tabular}{ | l || l |   }
    \hline
    №1 &  \hspace{10mm} \\ \hline
    №2 &  \\ \hline
    №3 &  \\ \hline
    №4 &  \\ \hline
    №5 &  \\ \hline
    №6 &  \\ \hline
    №7 &  \\ \hline
    №8 &  \\ \hline
    №9 &  \\ \hline
    №10 &  \\ \hline
    №11 &  \\ \hline
    \hline

    №12 &  \\ \hline
    №13 (a) & \\ \hline
    №13 (б) & \\ \hline
    №14 &  \\ \hline
    №15 &  \\ \hline
    №16 (а)&  \\ \hline
    №16 (б)&  \\ \hline
    №17 &  \\ \hline
    №18(а) &  \\ \hline
    №18(б) &  \\ \hline
    №18(в) &  \\ \hline
    \hline
    \end{tabular}\\
    
    
    \begin{tabular}{| l || l |}
        \hline
    $\sum$ (Тест ) & \\ \hline
        $\sum$ (Разв. часть) & \\ \hline
        $\sum$  &  \\ \hline
        \end{tabular}
    \end{multicols}
\end{landscape}
\newpage
\begin{landscape}
\begin{multicols}{2}
\subsubsection*{Тестовая часть:}
\noindent\boxed{№1} Найдите корень уравнения . \\


\noindent\boxed{№2} \vspace{30mm}

\noindent\boxed{№3} \texttt{Условие}
\vspace{10mm}
\begin{center}
    \texttt{Картинка}
    \vspace{20mm}
\end{center}


\noindent\boxed{№4} \texttt{Условие}\\


\noindent\boxed{№5} \texttt{Условие}\\
\vspace{20mm}
\begin{center}
    \texttt{Картинка}
    \vspace{20mm}
\end{center}
\noindent\boxed{№6} \texttt{Условие}

\columnbreak

\begin{center}
    \texttt{Картинка}
    \vspace{30mm}
\end{center}
\noindent\boxed{№7} \texttt{Условие}\\
    \vspace{35mm}


\noindent\boxed{№8} \texttt{Условие}\\
    \vspace{25mm}

\noindent\boxed{№9} На рисунке изображён график функции вида $f(x) = $, где числа a, b и c -- \texttt{какие-то}. Найдите значение $f(  )$.
\begin{center}
    \texttt{картинка}
    \vspace{30mm}
\end{center}
\noindent\boxed{№10} \texttt{Условие}

\end{multicols}
\end{landscape}
\newpage
\begin{landscape}
\begin{multicols}{2}



\noindent\boxed{№11} Найдите наименьшее значение функции
\[ y = \text{ \texttt{какая-то функция} }\]
\subsubsection*{Задания с развернутым ответом:}
\fbox{№12} а) Решите уравнение: \\
б) Найдите все корни уравнения, принадлежащие  отрезку  .\\


\noindent\boxed{№13} \texttt{Условие}\\

\vspace{12mm}
\noindentа) Докажите, что \\
б) Найдите \\


\noindent\boxed{№14} Решите неравенство: $ $.\\


\noindent\boxed{№15} \texttt{Условие}\\

\vspace{55mm}
\noindent\boxed{№16} \texttt{Условие}\\

\vspace{15mm}
\noindentа) Докажите, что\\
б) Найдите \\
\columnbreak\\
\boxed{№17} Найдите все значения параметра $a$, при каждои из которых:
\[ \texttt{ условие } \]
\texttt{дополнительное условие}.\\

\noindent\boxed{№18} \texttt{Условие}\\
а) \texttt{Условие}\\
б) \texttt{условие}\\
в) \texttt{Условие}
\end{multicols}
\end{landscape}
\end{document}