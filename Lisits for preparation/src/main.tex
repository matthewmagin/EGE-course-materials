\documentclass[12pt]{article}
% Packages
\usepackage[left=1.1cm,right=1.1cm,
    top=2cm,bottom=2cm,bindingoffset=0cm]{geometry}
\usepackage{tipa}
\usepackage{euscript}
\usepackage{mathrsfs}
%\usepackage{txfonts}
\usepackage{textcomp}
\usepackage{cmap}
\usepackage{enumitem}
\usepackage{mathtext}
\usepackage[T2A]{fontenc}
\usepackage[utf8]{inputenc}
\usepackage[english,russian]{babel}
\usepackage{amsfonts}
\usepackage{amsmath,amsfonts,amssymb,amsthm,mathtools} % AMS
\usepackage{icomma} % "Умная" запятая: $0,2$ --- число, $0, 2$ --- перечисление
\usepackage{fancybox,fancyhdr} %this packages provides fancy up and bottom of page
\usepackage{fancyhdr}
\pagestyle{fancy}
\usepackage{setspace}
%\полуторный интервал
%%% Колонтитулы
\usepackage{setspace}
\usepackage{fancyhdr}
\pagestyle{fancy}
\renewcommand{\sectionmark}[1]{\markright{\thesection\ #1}}

\fancyhead[LE,RO]{\thepage}
\fancyhead[LO]{\rightmark}
\fancyhead[RE]{\leftmark}
\usepackage{import}
\usepackage{xifthen}
\usepackage{pdfpages}
\usepackage{transparent}
\DeclareMathOperator{\arsec}{arsec}
\DeclareMathOperator{\arcsch}{arcsch}
\DeclareMathOperator{\arcosh}{arcosh}
\DeclareMathOperator{\arsinh}{arsinh}
\DeclareMathOperator{\artanh}{artanh}
\DeclareMathOperator{\arsech}{arsech}
\DeclareMathOperator{\arcoth}{arcoth}
\usepackage{moresize}
\usepackage{caption}
\usepackage{color}
\usepackage[colorlinks,urlcolor = blue, filecolor=blue,citecolor=blue, linkcolor = blue]{hyperref}
\begin{document}
\begin{titlepage}
    \begin{center}
        \vspace*{4cm}

        \HUGE
        \textsf{Сборник листочков}

        \vspace{0.5cm}
        \LARGE
        \textsf{Подготовка к ЕГЭ по профильной математике}

        \vspace{15cm}

        \textsf{Матвей Магин}

        \vfill








    \end{center}
\end{titlepage}
\tableofcontents
\newpage
\section{Тригонометрия}
	\subsection{Уравнения, сводящиеся к квадратным, уравнения на базовые формулы}
\subsubsection*{Уравнения на применение базовых формул и сведение к квадратным уравнениям}
\begin{enumerate}[start=1,label={\itshape\bfseries \arabic*.}]
\item  а) Решите уравнение: $\cos{2x} - 3\cos{x} + 2 = 0$\\
  б) Найдите все корни, принадлежащие промежутку $[-4\pi;- 5\pi / 2]$.
\item а) (\textit{Досрочная волна 2015}) Решите уравнение: $\sin{2x} +\sqrt{2}\sin{x} = 2\cos{x} + \sqrt{2}$ \\
  б) Найдите все корни, принадлежащие промежутку $[\pi; 5\pi / 2]$.

\item  а) Решите уравнение: $2\sin{\bigg(\cfrac{3\pi}{2} - x\bigg)} \cdot \cos{\bigg(x + \cfrac{\pi}{2}\bigg)} = \sqrt{3} \cos{(2\pi - x)} $\\
  б) Найдите все корни, принадлежащие промежутку $[-2\pi; -\pi]$.

\item  а) Решите уравнение: $\sin{8\pi x} + 1 = \cos{4\pi x} + \sqrt{2} \cos{(4\pi x - \frac{\pi}{4})}$\\
  б) Найдите все корни, принадлежащие промежутку $[2 - \sqrt{7};\sqrt{7} - 2]$.

\item  а) Решите уравнение: $4\cos^4{x} - 4\cos^2{x} + 1 = 0$\\
  б) Найдите все корни, принадлежащие промежутку $[-2\pi;-\pi]$.

\item  а) Решите уравнение: $4\sin^4{2x} + 3\cos{4x} -1 = 0$\\
  б) Найдите все корни, принадлежащие промежутку $[\pi; 3\pi / 2]$.

\item  а) Решите уравнение: $\sin{\frac{7x}{2}}\sin{\frac{x}{2}} + \cos{\frac{7x}{2}}\cos{\frac{x}{2}} = \cos^3{3x}$\\
б) Найдите все корни, принадлежащие промежутку $[\pi; 3\pi / 2]$.

\item  а) Решите уравнение: $\tg^2{x} + (1 + \sqrt{3})\tg{x} + \sqrt{3} = 0$\\
б) Найдите все корни, принадлежащие промежутку $[- 5\pi / 2; 4\pi]$.

\item а) Решите уравнение: $\sqrt{2} \sin^3{x} - \sqrt{2} \sin{x} + \cos^2{x} = 0$\\
б) Найдите все корни, принадлежащие промежутку $[ - 5\pi / 2; -\pi]$.

\item  а) Решите уравнение: $\cos^2{(\pi - x)} - \sin{(x + \frac{3\pi}{2})} = 0 $\\
  б) Найдите все корни, принадлежащие промежутку $[3\pi; 9\pi / 2]$.

\item  а)(*) Решите уравнение: $\cos^6{x} + \sin^6{x} = \frac{1}{4} \sin^2{2x}$\\  б) Найдите все корни, принадлежащие промежутку $(- 3\pi /4; \pi / 2]$.

\item  а)(**) Решите уравнение: $\sin^3{x} + \cos^3{x} = \sin^2{x} + \cos^2{x}$\\
б) Найдите все корни, принадлежащие промежутку $[9\pi / 4; 7\pi / 2]$.
\end{enumerate}
\textit{Если совсем не получаются номера со звездочкой:}\\

\noindent$(*)$ Вспомнить формулу $x^3 + y^3 = (x + y)(x^2 - xy + y^2)$. Вспомнить, что $\sin^4{x} + \cos^4{x} = (\cos^2{x} + \sin^2{x})^2 - 2\cdot\sin^2{x}\cos^2{x} = 1 - 2\cdot\sin^2{x}\cos^2{x}$\\
\\
$(**)$ Вспомнить формулу суммы кубов (да, да), посмотреть, что из себя представляет  скобка $(\sin^2{x} - \sin^2{x} \cos^2{x} + \cos^2{x})$.



%%%%% КОНЕЦ ПЕРВОГО ЛИСТОЧКА


\subsection{Нестандартные методы: метод вспомогательного угла, метод оценки, универсальная подстановка}

\begin{enumerate}[start=1,label={\itshape\bfseries \arabic*.}]

\item а) Решите уравнение: $\cos^2{2x} + \cos^2{3x} = 1$\\
      б) Найдите все корни уравнения, принадлежащие отрезку $[5\pi / 2; 4\pi]$

\item а) Решите уравнение: $\sin{x} + \sin{2x} + \sin{3x} + \sin{4x} = 0$\\
      б) Найдите все корни уравнения, принадлежащие отрезку $[- 3\pi / 2; 0]$

\item а) Решите уравнение: $\sin^4{x} + \cos^4{x} = \sin{2x} - \cfrac{1}{2}$\\
      б) Найдите все корни уравнения, принадлежащие отрезку $[-\pi; - 5\pi / 2]$

\item а) Решите уравнение: $15\cos{x} - 8\sin{x} = 8,5$\\
      б) Найдите все корни уравнения, принадлежащие отрезку $[\pi; 5\pi / 2]$

\item а) Решите уравнение: $\sin{3x} + \sin^3{x} = \cfrac{3\sqrt{3}}{4}\sin{2x}$\\
      б) Найдите все корни уравнения, принадлежащие отрезку $[5\pi / 2; 4\pi]$

\item а) Решите уравнение: $\cos{x}(2\cos^2{x} - 1) = \cfrac{1}{4}$\\
      б) Найдите все корни уравнения, принадлежащие отрезку $[\pi / 2; 2\pi]$

\item а) Решите уравнение: $3 + \sin{3x}\sin{x} = 3\cos{2x}$\\
      б) Найдите все корни уравнения, принадлежащие отрезку $[\pi / 2; 3\pi]$

\item а) Решите уравнение: $\cos^5{x} + \sin^4{x} = 1$\\
      б) Найдите все корни уравнения, принадлежащие отрезку $[-9\pi / 2; -3\pi]$

\item а) Решите уравнение: $\cos{x} + \cos{2x} + \cos{4x} = 0$\\
      б) Найдите все корни уравнения, принадлежащие отрезку $[-11\pi / 2; -4\pi]$

\item а) Решите уравнение: $\sin^4{x} + 12\cos^2{x} = 7$\\
      б) Найдите все корни уравнения, принадлежащие отрезку $[-5\pi / 2; -\pi]$

\item а) Решите уравнение: $2\sin{x}\cos{x} - 6(\sin{x} - \cos{x}) + 6 = 0$\\
      б) Найдите все корни уравнения, принадлежащие отрезку $[7\pi / 2; 5\pi]$

\item а) Решите уравнение: $\sin{x} + \sin^2{x} + \sin^3{x} = \cos{x} + \cos^2{x} + \cos^3{x}$\\
      б) Найдите все корни уравнения, принадлежащие отрезку $[-4\pi; -5\pi / 2]$
\end{enumerate}


%%%%% КОНЕЦ ВТОРОГО ЛИСТОЧКА




\subsection{Уравнения на учёт ОДЗ, тригонометрические неравенства}

\begin{enumerate}[start=1,label={\itshape\bfseries \arabic*.}]

\item а)\textit{(Основная волна, 2016)} Решите уравнение: $\cfrac{\sin{2x}}{\sin{\bigg(\cfrac{7\pi}{2} - x\bigg)}} = \sqrt{2}$ \\
      б) Найдите все корни уравнения, принадлежащие отрезку $[2\pi; 7\pi / 2]$

\item а) Решите уравнение: $\cfrac{2\sin^2{x} + 2\sin{x}\cos{x} - 1}{\sqrt{\cos{x}}} = 0$\\
      б) Найдите все корни уравнения, принадлежащие отрезку $[-4\pi; -5\pi / 2]$

\item а) Решите уравнение: $\tg{x} (\ctg{x} - \cos{x}) = 2\sin^2{x}$\\
      б) Найдите все корни уравнения, принадлежащие отрезку $[-5\pi / 2; -\pi]$

\item а) Решите уравнение: $\tg{5x} = \tg{x}$\\
      б) Найдите все корни уравнения, принадлежащие отрезку $[-7\pi / 2; -2\pi]$

\item а) Решите уравнение: $(2\cos^2{x} + \sin{x} - 2) \sqrt{5\tg{x}} = 0$\\
      б) Найдите все корни уравнения, принадлежащие отрезку $[\pi; 5\pi / 2]$

\item а) Решите уравнение: $\cfrac{26\cos^2{x} - 23\cos{x} + 5}{13\sin{x} - 12} = 0$\\
      б) Найдите все корни уравнения, принадлежащие отрезку $[-5\pi / 2; -\pi]$

\item а) Решите уравнение: $\cfrac{26\cos^2{x} - 23\cos{x} + 5}{13\sin{x} - 12} = 0$\\
      б) Найдите все корни уравнения, принадлежащие отрезку $[-5\pi / 2; -\pi]$

\item а) Решите уравнение: $4\tg^2{x} + \cfrac{11}{\sin{\bigg(\cfrac{3\pi}{2} - x\bigg)}} + 10 = 0$\\
      б) Найдите все корни уравнения, принадлежащие отрезку $[-7\pi / 2; -2\pi]$

\item а) Решите уравнение: $2\sin{2x} - \sin{x} \cdot \sqrt{2\ctg{x}} = 1$\\
      б) Найдите все корни уравнения, принадлежащие отрезку $[0; \pi]$

\item а) Решите уравнение: $\sqrt{\sin{x} - \cos{x}} \cdot (\ctg{x} - \sqrt{3}) = 0$\\
      б) Найдите все корни уравнения, принадлежащие отрезку $[3\pi / 2; 3\pi]$
\end{enumerate}

\noindent\underline{\textit{Уравнения, в которых не обязательно есть учет ОДЗ, но их хорошо порешать:}}

\begin{enumerate}[start=1,label={\itshape\bfseries \arabic*.}]

\item а) Решите уравнение: $\cos^2{x} + \cos^2{2x} + \cos^2{3x} + \cos^2{4x} = 2$\\
      б) Найдите все корни уравнения, принадлежащие отрезку $[\pi / 2; 2\pi]$

\item а) Решите уравнение: $\sin{x} \cdot \sqrt{3 - \tg^2{\cfrac{3x}{2}}} - \cos{x} = 2$\\
      б) Найдите все корни уравнения, принадлежащие отрезку $[-17; 2]$

\item а) Решите уравнение: $\cos^2{x} + \cos^2{2x} = \sin^2{3x} + \sin^2{4x}$\\
      б) Найдите все корни уравнения, принадлежащие отрезку $[3\pi / 2; 3\pi]$

\end{enumerate}

\noindent\underline{\textit{Тригонометрические неравенства:}}

\begin{enumerate}[start=1,label={\itshape\bfseries \arabic*.}]

\item Решите неравенство: $\cfrac{2\cos^2{x} - \cos{x} - 1}{\sqrt{3} \tg{x} + 1} \le 0$

\item Решите неравенство: $\cfrac{\cos{2x} + 3\cos{x} + 2}{(\sin{x} - 1)\sin{x}} \ge 0$

\item Решите неравенство: $\cfrac{(\sin{x} - 1)(2\cos{x} + \sqrt{3})}{2\sin{x} + 3} \ge 0$

\end{enumerate}


%%%%% КОНЕЦ ТРЕТЬЕГО ЛИСТОЧКА


        \section{Стереометрия}
        \subsection{Аналитическая геометрия и метод координат}

    \begin{enumerate}[start=1,label={\itshape\bfseries \arabic*.}]
        \item  В кубе $ABCD A_1 B_1 C_1 D_1$ на рёбрах $A_1 B_A$, $B_1 C_1$, $AD$ выбраны точки $K, M, N$ соответственно так, что $A_1 K : K B_1 = C_1 M : M B_1 = DN : NA = 1 : 2$.\\
        а) Докажите, что прямая $B D_1$ перпендикулярна плоскости $(KMN)$.\\
        б) Найдите расстояние от точки $A$ до плоскости $(KMN)$, если ребро куба равно 6.
        \item В кубе $ABCD A_1 B_1 C_1 D_1$ со стороной 8 на ребре $A A_1$ взята точка $K$ такая, что $A_1 K = 1$. Через точки $K$ и $B_1$ проведена плоскость $\alpha$, параллельная прямой $(A C_1)$.\\
        а) Докажите, что $A_1 P : PD = 1 : 6$, где $P$ -- точка пересечения плоскости $\alpha$ и ребра $A_1 D_1$.\\
        б) Найдите угол между плоскостью $\alpha$ и плоскостью $(A D D_1)$.
        \item \textit{(Основная волна, 2021г.)}\\
        В правильной прямоугольной пирамиде $SABCD$ сторона основания $AD = 14$, высота $SH = 24$. Точка $K$ -- середина бокового ребра $SD$, а точка $N$ -- середина ребра $CD$. Плоскость $(ABK)$ пересекает боковое ребро $SC$ в точке $P$.\\
        а) Докажите, что прямая $(KP)$ пересекает отрезок $SN$ в его середине.\\
        б) Найдите расстояние от точки $P$ до плоскости $(ABS)$.
        \item Точка $K$ лежит на стороне $AB$ основания $ABCD$ правильной четырехугольной пирамиды $SABCD$, все рёбра которой равны. Плоскость  $\alpha$ проходит через точку $K$ параллельно плоскости $(ASD)$. Сечение пирамиды плоскостью $\alpha$ -- четырехугольник, в который можно вписать окружность. \\
        а) Докажите, что $BK = 2AK$.\\
        б) Найдите расстояние от вершины $S$ до плоскости $\alpha$, если все рёбра пирамиды равны 1.
        \item \textit{(Основная волна, 2019г.)}\\
        В правильной треугольной пирамиде $SABC$ сторона основания $AB$ равна 9, а боковое ребро $SA = 6$. На рёбрах $AB$ и $SC$ отмечены точки $K$ и $M$ соответственно, причем $AK : KB = SM :  MC = 2 : 7$. Плоскость $\alpha$ содержит прямую $(KM)$ и параллельна прямой $(SA)$.\\
        а) Докажите, что плоскость $\alpha$ делит ребро $(SB)$ в отношении $2 : 7$.\\
        б) Найдите расстояние между прямыми $(SA)$ и $(KM)$.
        \item В правильной четырехугольной пирамиде $PABCD$ проведена высота $PH$. Точка $N$ -- середина отрезка $AH$, а точка $M$ -- середина ребра $AP$.\\
        а) Докажите, что угол между прямыми $(PH)$ и $(BM)$ равен углу $\angle BMN$.\\
        б) Пусть длины всех рёбер данной пирамиды равны между собой. Найдите угол между прямыми $(PH)$ и $(BM)$.
        \item В основании прямой треугольной призмы $ABC A_1 B_1 C_1$ лежит прямоугольный треугольник $ABC$ с прямым углом $C$, $AC = 4$, $BC = 16$, $AA_1 = 4\sqrt{2}$. Точка $Q$ -- середина ребра $A_1 B_1$, а точка $P$ делит ребро $B_1 C_1$ в отношении $1 : 2$, считая от вершины $C_1$. Плоскость $APQ$ пересекает ребро $C C_1$ в точке $M$.\\
        а) Докажите, что точка $M$ -- середина ребра $C C_1$.\\
        б) Найдите расстояние от точки $A_1$ до плоскости $APQ$.
    \end{enumerate}

    \subsection{Объемы и объемные отношения}
    \begin{enumerate}[start=1,label={\itshape\bfseries \arabic*.}]
        \item В основании прямой треугольной призмы $ABC A_1 B_1 C_1$ лежит прямоугольный треугольник $ABC$ с прямым углом $C$, $AC = 4$, $BC = 16$, $AA_1 = 4\sqrt{2}$. Точка $Q$ -- середина ребра $A_1 B_1$, а точка $P$ делит ребро $B_1 C_1$ в отношении $1 : 2$, считая от вершины $C_1$. Плоскость $APQ$ пересекает ребро $C C_1$ в точке $M$.\\
        а) Докажите, что точка $M$ -- середина ребра $C C_1$.\\
        б) Найдите расстояние от точки $A_1$ до плоскости $APQ$.
        \item В треугольной пирамиде $ABCD$ двугранные углы при рёбрах $AD$ и $BC$ равны. $AD = BD = DC = AC = 5$.\\
        а) Докажите, что $AD = BC$.\\
        б) Найдите объем пирамиды, если двугранные углы при $AD$ и $BC$ равны $60^{\circ}$.
        \item Дана правильная треугольная призма $ABC A_1 B_1 C_1$, в которой $AB = 6$ и $A A_1 = 3$. Точки $O$ и $O_1$ соответственно являются центрами окружностей, описанных около треугольников $ABC$ и $A_1 B_1 C_1$ соответственно. На ребре $C C_1$ отмечена точка $M$ так, что $CM = 1$.\\
        а) Докажите, что прямая $(O O_1)$ содержит точку пересечения медиан треугольника $ABM$.\\
        б) Вычислите объем пирамиды $ABM C_1$.
        \item Дана пирамида $PABCD$, в основнии -- трапеция $ABCD$ с большим основанием $AD$. Известно, что сумма углов $\angle BAD$ и $\angle ADC$ равна $90^{\circ}$, а плоскости $(PAB)$ и $(PCD)$ перпендикулярны основанию, прямые $(AB)$ и $(CD)$ пересекаются в точке $K$.\\
        а) Докажите, что плоскость $(PAB)$ перпендикулярна плоскости $(PCD)$.\\
        б) Найдите объем  $PKBC$, если известно, что $AB = BC = CD = 2$, а $PK = 12$.
    \end{enumerate}

    \subsection{Тела вращения}

        \begin{enumerate}[start=1,label={\itshape\bfseries \arabic*.}]
        \item \textit{(Основная волна, 2018г.)}\\
        В цилиндре образующая перпендикулярна плоскости основания. На окружности одного из оснований цилиндра выбраны точки $A$, $B$ и $C$ так, а на окружности другого основания -- $C_1$, причем $C C_1$ -- образующая цилиндра, а $AC$ -- диаметр основания. Известно, что $\angle ACB = 30^{\circ}, \ AB = \sqrt{2}, \ C C_1 = 2$.\\
        а) Докажите, что угол между прямыми $(AC_1)$ и $(BC)$ равен $45^{\circ}$.\\
        б) Найдите объем цилиндра.
        \item Точки $A$, $B$ и $C$лежат на окружности основания конуса с вершиной $S$, причем $A$ и $C$ диаметрально противоположны. Точка $M$ -- середина $BC$.\\
        а) Докажите, что прямая $(SM)$ образует с плоскостью $(ABC)$ такой же угол, как прямая $(AB)$ с плоскостью $(SBC)$.\\
        б) Найдите угол между прямой $(SA)$ и плоскостью $(SBC)$, если $AB = 6$, $BC = 8$, и $AS = 5\sqrt{2}$.
        \item Радиус основания конуса с вершиной $S$ и центром основания $O$ равен 5, а его высота равна $\sqrt{51}$. Точка $M$ -- середина образующей $SA$, конуса, а точки $N$ и $B$ лежат на основании конуса, причем прямая $(MN)$ параллельна образующей конуса $(SB)$.\\
        а) Докажите, что угол $\angle ANO$ прямой. \\
        б) Найдите угол между прямой $(BM)$ и плоскостью основания конуса, если $AB = 8$.
        \item Шар проходит через вершины одной грани куба и касается сторон противоположной грани куба.\\
        а) Докажите, что сфера касается рёбер в их серединах.\\
        б) Найдите объем шара, если ребро куба равно 1.
     \end{enumerate}


%%%%%%%%%%%%%%%%%%%%%%%%%%%%


\section{Показательные уравнения и неравенства}
\subsection{Базовые свойства степени и стандартные преобразования, базовые методы решения}

\subsubsection*{Уравнения:}

\begin{enumerate}[start=1,label={\itshape\bfseries \arabic*.}]

\item а) Решите уравнение: $\sqrt{3^x} \cdot 5^{\frac{x}{2}} = 225$\\
      б) Найдите все корни уравнения, принадлежащие отрезку $[\frac{3\sqrt{37}}{8}; \frac{5\sqrt{43}}{6}] $

\item а) Решите уравнение: $4^x - \cfrac{5}{4} \cdot 2^{x + 2} + 4 = 0 $\\
      б) Найдите все корни уравнения, принадлежащие отрезку $(0; 4^{\frac{\sqrt{2}}{2}}]$

 \item а) Решите уравнение: $5^{2x + 1} - 3 \cdot 5^{2x - 1} -550 = 0$\\
       б) Найдите все корни уравнения, принадлежащие отрезку $[\frac{4\sqrt{39}}{17}; \frac{7\sqrt{19}}{9}]$


\item а) Решите уравнение: $4^{x + \sqrt{x^2 -2 }} - 5 \cdot 2^{x - 1 + \sqrt{x^2 - 2}} = 6$\\
      б) Найдите все корни уравнения, принадлежащие отрезку $[\frac{5\sqrt{17}}{14}; \frac{3\sqrt{29}}{5}]$

\item а) Решите уравнение: $\sqrt{3^{x - 54}} - 7 \cdot \sqrt{3^{x - 58}} = 162$\\
      б) Найдите все корни уравнения, принадлежащие отрезку $[2^{\sqrt{33}};2^{\sqrt{66}} ]$

\item а) Решите уравнение: $2^{2x^2 - 5x - 1} = 0,5 \cdot \sqrt[3]{4^{2x}}$\\
      б) Найдите все корни уравнения, принадлежащие отрезку $[239^{-239}; \sqrt{53}]$

\item а) Решите уравнение: $3^{x + 1} - 5^x + 3^{x - 1} - 5^{x - 1} = 5^{x - 2} - 3^{x - 2}$\\
      б) Найдите все корни уравнения, принадлежащие отрезку $[0; \frac{\sqrt{239}}{30}]$

\end{enumerate}

\subsubsection*{Неравенства:}

\begin{enumerate}[start=1,label={\itshape\bfseries \arabic*.}]

\item Решите неравенство: $\left(\cfrac{\sqrt{239}}{6}\right)^{\frac{x^2 - 2x - 3}{x^2 - 2x + 1}} \ge 1$

\item Решите неравенство: $7^x - 7^{1 - x} + 6 > 0$

\item Решите неравенство: $3^{-2x + 4} - 81 \cdot 3^{-x + 3} - 3^{-x + 1} + 81 \le 0$

\item Решите неравенство: $64^{x^2 - 3x + 20} - 0,125^{2x^2 - 6x - 200} \le 0$

\item (\textit{Основная волна, 2018г.}) Решите неравенство: $\cfrac{3^x + 9}{3^x - 9}  + \cfrac{3^x - 9}{3^x + 9} \ge \cfrac{4 \cdot 3^{x + 1} + 144}{9^x - 81}$

\item Решите неравенство: $4^{x^2 + 2x + 5} - 33 \cdot 2^{x^2 + 2x + 5} + 8 \ge 0$

\end{enumerate}

%%%%% КОНЕЦ ЧЕТВЕРТОГО ЛИСТОЧКА





\subsection{Более сложные методы, однородные и смешанные уравнения и неравенства}

\subsubsection*{Уравнения:}

\begin{enumerate}[start=1,label={\itshape\bfseries \arabic*.}]

\item а) Решите уравнение: $27^x + 12^x = 2 \cdot 8^x$\\
      б) Найдите все корни уравнения, принадлежащие отрезку $[0; 239^{\sqrt{239}}]$

\item а) Решите уравнение: $9^{\sin{x}} + 9^{-\sin{x}} = \cfrac{10}{3}$\\
      б) Найдите все корни уравнения, принадлежащие отрезку $[5\pi. 2; 4\pi]$

 \item а) Решите уравнение: $3^{2x^2 + 6x - 9} + 4\cdot15^{x^2 + 3x - 5} = 3 \cdot 5^{2x^2 + 6x - 9}$\\
       б) Найдите все корни уравнения, принадлежащие отрезку $[-\sqrt{2}^3; \sqrt{3}^{\sqrt{2}}]$

 \item а) Решите уравнение: $(25^{\sin{x}})^{\cos{x}} = 5^{\sqrt{3}\sin{x}}$\\
       б) Найдите все корни уравнения, принадлежащие отрезку $[\frac{5\pi}{2}; 4\pi]$

\item а) Решите уравнение: $\cfrac{3^{\cos^2{x}} + 3^{\sin^2{x}} - 4}{\sin{x} + 1} = 0$\\
      б) Найдите все корни уравнения, принадлежащие отрезку $[\frac{11\pi}{2}; 7\pi]$

\item а) Решите уравнение: $256^{\sin{x}\cos{x}} - 18 \cdot 16^{\sin{x}\cos{x}} + 32 = 0$\\
      б) Найдите все корни уравнения, принадлежащие отрезку $[\frac{9\pi}{2}; 6\pi]$
 \end{enumerate}

 \subsubsection*{Неравенства:}

 \begin{enumerate}[start=1,label={\itshape\bfseries \arabic*.}]

\item Решите неравенство: $\cfrac{13 - 5 \cdot 3^x}{9^x - 12 \cdot 3^x + 27} \ge 0,5$

\item Решите неравенство: $2^{\frac{x}{x + 1}} - 2^{\frac{5x + 3}{x + 1}} + 8 \le 2^{\frac{2x}{x + 1}}$

\item Решите неравенство: $\cfrac{3}{(2^{2 - x^2})^2} - \cfrac{4}{2^{2 - x^2} - 1} + 1 \ge 0$

\item Решите неравенство: $\cfrac{2 \cdot 3^{2x + 1} - 7 \cdot 6^x + 2 \cdot 4^x}{3 \cdot 9^x - 3^x \cdot 2^{x + 1 }} \le 1$

\item Решите неравенство: $\cfrac{2^{2x + 1} - 96 \cdot 0,5^{2x + 3} + 2}{x + 1} \le 0$

\end{enumerate}

%%%%% КОНЕЦ ПЯТОГО ЛИСТОЧКА



\newpage
\section{Иррациональные уравнения и неравенства}
\subsection{Повторение и основные равносильные переходы}

\subsubsection*{Уравнения:}
\begin{enumerate}[start=1,label={\itshape\bfseries \arabic*.}]

\item а) (\textit{Основная волна 2018г;  резервный день}) Решите уравнение: $x - 3\sqrt{x - 1} + 1 = 0$ \\
      б) Найдите все корни уравнения, принадлежащие отрезку $[\sqrt{3}; \sqrt{20}]$

\item а) Решите уравнение: $\sqrt{3x^2 + 5x - 2} = 3x - 1$\\
      б) Найдите все корни уравнения, принадлежащие отрезку $[-2\sqrt{3}; \sqrt{10}]$

\item а) Решите уравнение: $2x^2 + 3x - 5\sqrt{2x^2 +3x + 9} + 3 = 0$\\
      б) Найдите все корни уравнения, принадлежащие отрезку $[\cfrac{1}{\sqrt{5}}; \sqrt{2}]$

\item а) Решите уравнение: $\sqrt{x - 2} = 3 - \sqrt{11 - x}$\\
      б) Найдите все корни уравнения, принадлежащие отрезку $[\cfrac{\sqrt{17}}{2}; \sqrt{122}]$

\item а) Решите уравнение: $\sqrt{3x^2 - 2x + 15} + \sqrt{3x^2 - 2x + 8} = 7$\\
      б) Найдите все корни уравнения, принадлежащие отрезку $[-\cfrac{1}{\sqrt{5}}; \sqrt{3}]$

\item а) Решите уравнение: $\sqrt{x^2 - x - 2} + \sqrt{3x^2 + x - 2} = 0$\\
      б) Найдите все корни уравнения, принадлежащие отрезку $[-\cfrac{\sqrt{5}}{2}; 1]$

\end{enumerate}

\subsubsection*{Неравенства:}

\begin{enumerate}[start=1,label={\itshape\bfseries \arabic*.}]

\item Решите неравенство: $\sqrt{\cfrac{x + 2}{2x + 1}} \ge x$

\item Решите неравенство: $\cfrac{\sqrt{3x + 10 - x^2}}{\sqrt{x + 6} - x} \ge 0$

\item Решите неравенство: $\sqrt{\cfrac{x + 3}{x - 3}} + 2\sqrt{\cfrac{x - 3}{x + 3}} < 3$

\item Решите неравенство: $\cfrac{\sqrt{x^2 - 3x + 2}}{x - 2} \le 2$

\item Решите неравенство: $\sqrt{2x + \sqrt{6x^2 + 1}} \le x + 1$

\item Решите неравенство: $\sqrt{8 - x} + \sqrt{2x + 8} > \sqrt{5x + 16}$

\item Решите неравенство: $\sqrt{4x + 1} > \cfrac{x + 4}{x}$

\end{enumerate}

%%%%% КОНЕЦ ШЕСТОГО ЛИСТОЧКА

\newpage

\section{Логарифмические неравенства}


\subsection{Преобразования и замены}
 \begin{enumerate}[start=1,label={\itshape\bfseries \arabic*.}]
    \item $\log_2{(x - 1)(x^2 + 2)} \le 1 + \log_2{(x^2 + 3x - 4)} - \log_2{x}$
    \item $2\log_2{(1 - 2x) - \log_2\left( \cfrac{1 - 2x}{x} \right)} \le \log_2{(4x^2 + 6x - 1)}$
    \item $\log_{\sqrt{2}}{x} - \log_{8x^2}{4} \le -2$
    \item $7\log_3{(x^2 - 7x + 12)} \le 8 + \log_3{\cfrac{(x - 3)^7}{x - 4}}$
    \item $1 - \cfrac{1}{2}\log_{\sqrt{3}}{\cfrac{x + 9}{x + 3}} \ge \log_9{(x + 1)^2}$
    \item $\log_{\frac{3}{4}}{\log_6{\cfrac{x^2 + x}{x + 4}}} < 0$
    \item $\cfrac{\log_2{8x} \cdot \log_{0,125x}{2}}{\log_{0,25x}{16}} \le \cfrac{1}{4}$
    \item $\cfrac{2}{\log_2{(2x - 2)}} + \cfrac{3}{\log_2{(4x - 4)}} \le \cfrac{8}{\log_3{27} + \log_2{(x - 1)}}$
\end{enumerate}
\subsection{Метод рационализации и преобразования}
 \begin{enumerate}[start=1,label={\itshape\bfseries \arabic*.}]
    \item $\log_{6x}{x} \cdot \log_{3x}{x} > 0$
    \item $\log_{x + 1}{(5x^2 - x)} \ge 2$
    \item $\log_x{(3 - x)} \cdot \log_{2x - 1}{(3 - x)} < \log_x{(3 - x)} \cdot \log_{5 - 2x}{(3 - x)}$
    \item $\cfrac{\log_{2x - 1}\log_2{(x^2 - 2x)}}{\log_{2x - 1}{(x^2 - 6x + 10)}} \le 0$
    \item $\log_{|x|}{(\sqrt{9 - x^2} - x - 1)} \ge 1$
    \item $\log_{(x + 2)^2}{(x(x + 1)(x + 3)(x + 4))} > 1$
    \item $\log_x{(x + 2)} + 2\log_{x + 2}{x} > 3$
    \item $\cfrac{\log_{2^{(x + 1)^2 - 1}}{\log_{2x^2 + 2x + 3}{(x^2 - 2x)}}}{\log_{2^{(x + 1)^2} - 1}{(x^2 + 6x + 10)}} \ge 0$
\end{enumerate}
\subsection{Метод интервалов}
 \begin{enumerate}[start=1,label={\itshape\bfseries \arabic*.}]
    \item $\log_2{(2^x - 1)} \cdot \log_{\frac{1}{2}}{(2^{x + 1} - 2)} > -2$
    \item $\cfrac{x - 1}{\log_3{(9 - 3^x)} - 3} \le 1$
    \item $\cfrac{\log_2{(4^{x + 1} + 2)} - 1}{x + 1} > 2$
    \item $\log_{4x}{2x} \cdot \log^2_{x}{4} \le \log_x{\sqrt[4]{x^3}}$
    \item $\log_x{3} \cdot \log_{3x}{9} > 1$
\end{enumerate}
\subsection{Дополнительные задачи на разные темы}
\begin{enumerate}[start=1,label={\itshape\bfseries \arabic*.}]
    \item $\log_3(x^4 - 4x^2 + 4) + 3 \ge \log_9(2 - x^2)$
    \item $2^x \lg(x^2 - 1) + 3\lg(x + 5) \ge 3\lg(x^2 - 1) + 2^x \lg(x + 5)$
    \item $2x \ge \log_2\bigg( \cfrac{35}{3} \cdot 6^{x - 1} - 2 \cdot 9^{x - \frac{1}{2}} \bigg)$
    \item $\cfrac{\sqrt{4 - x} - \sqrt{x^3 - 5x^2 + 4x}}{\sqrt{4 - x} + \log_{5x + 1}^{2}(x^3 - 5x^2 + 4x + 1)} \ge 1$
    \item $\log_{9}^{2}(729 - x^2) - 5\log_{9}(729 - x^2) + 6 \ge 0$
    \item $\log_{1 - \frac{1}{(x - 1)^2}}\bigg( \cfrac{x^2 + 5x + 8}{x^2 - 3x + 2} \bigg) \le 0$
    \item $\cfrac{\log_{5}(7x^2 - 10x + 4)}{\log_{5}{x}} > 2$
    \item $3^{\log_2{x^2}} + 2 \cdot |x|^{\log_2{9}} \le 3 \cdot \bigg( \cfrac{1}{3} \bigg)^{\log_{0,5}(2x + 3)}$
    \item $\log_{x}(x + 1) \cdot \log_{x + 1}(x + 2) \cdot \log_{x + 2}(x + 3) \cdot \log_{x + 3}(x + 4) \cdot \log_{x + 4}(x + 5) \cdot \log_{x + 5}(x + 6) \le 2$
    \item $\log_{5x + 2}(\log_{8 - x}(x + 2)) \ge 0$
    \item $\log_3{(3x + 1)} + \log_5{\bigg( \cfrac{1}{72x^2} + 1\bigg)} \ge \log_5{(\frac{1}{24x} + 1)}$
    \item $\log_{1 - \frac{x^2}{37}} (x^2 - 12|x| + 37) - \log_{1 + \frac{x^2}{37}}(x^2 - 12|x| + 37) \ge 0$
    \item $\log_{x}(\sqrt{x^2 + 2x - 3} + 2) \cdot \log_{5}(x^2 + 2x - 2) \ge \log_x{4}$
    \item $\log_3{(x - 1)} \cdot \log_3(3^{x + 1} + 3) \cdot \log_{x - 1}(3^x + 1) \ge 6$
\end{enumerate}

%%%%% КОНЕЦ СЕДЬМОГО ЛИСТОЧКА

\section{Экономические задачи}
\subsection{Кредиты}
\begin{enumerate}[start=1,label={\itshape\bfseries \arabic*.}]

    \item \emph{Аннуитетный платёж:}\\
    В июле 2018г. планируется взять кредит в банке. Условия его возврата таковы:
    \begin{itemize}
    \item каждый январь долг увеличивается на $20\%$ по сравнению с концом предыдущего года
    \item с февраля по июнь каждого года необходимо выплатить единым платежом часть долга
    \end{itemize}
    Сколько рублей было взято в долг в банке, если известно, что кредит был погашен четырьмя \emph{равными} платежами и банку было выплачено 311040 рублей.
    \item \emph{Дифференцированный платёж:}\\
    В июле планируется взять кредит в банке на сумму 8 млн. рублей на 10 лет. Условия его возврата таковы:
    \begin{itemize}
        \item каждый январь долг возрастает на $15\%$ по сравнению с концом предыдущего года
        \item с февраля по июнь каждого года необходимо выплатить часть долго
        \item в июле каждого года долг должен быть \emph{на одну и ту же величину} меньше долга на июнь предыдущего года
    \end{itemize}
    Сколько миллионов рублей составит общая сумма выплат после погашения кредита?
    \item \emph{Выплаты, заданные таблицей:}\\
    15-го января планируется взять кредит в банке на шесть месяцев в размере 1.5 млн. рублей. Условия его возврата таковы:
    \begin{itemize}
        \item 1-го числа каждого месяца долг увеличивается на $r$ процентов по сравнению с концом предыдущего месяца, где $r$ -- целое число.
        \item со 2-го по 14-е число каждого месяца необходимо выплатить часть долга
        \item 15-го числа каждого месяца долг должен составлять некоторую сумму в соответствии со следующей таблицей:\\
        \begin{tabular}{ | l | l | l | l | l | l | l | l | }
        \hline
        \textbf{Дата} & 15.01 & 15.02 & 15.03 & 15.04 & 15.05 & 15.06 & 15.07\\ \hline
        Долг (в млн. рублей) & 1,5 & 1,2 & 1 & 0,7 & 0,5 & 0,3 & 0\\
        \hline
        \end{tabular}
    \end{itemize}
    Найдите наименеьшее $r$ при условии, что общая сумма выплат больше, чем $2,2$млн. рублей

    \item \emph{Смешанные платежи:}\\
    В июле 2016 года Пётр взял в кредит сумму $S$ тысяч рублей на 5 лет. Условия выплаты долга таковы:
    \begin{itemize}
        \item каждый январь долг увеличивается на $20\%$
        \item с февраля по июнь должна быть выплачена часть долга
        \item известно, что в июле 2017, 2018,  2019 годов долг составлял $S$ тысяч рублей
        \item в июле 2020, 2021 годов было выплачено по 360 тысяч рублей.
        \item в июле 2021 года кредит был погашен
    \end{itemize}
    Сколько составляет общая сумма выплат?
    \item \emph{Смешанные платежи:}\\
    В июле 2025 года планируется взять кредит в банке на сумму 700 тысяч рублей на 10 лет. Условия его возврата таковы:
    \begin{itemize}
        \item в январе 2026, 2027, 2028, 2029 и 2030 годов долг возрастает на 19$\%$ по сравнению с концом предыдущего года
        \item в январе 2031, 2032, 2033, 2034 и 2035 годов долг возрастает на 16$\%$ по сравнению с концом предыдущего года
        \item с февраля по июнь каждого года необходимо выплатить часть долга
        \item в июле каждого года долг должен быть на одну и ту же величину меньше долга на июль предыдущего года
        \item к июлю 2035 года кредит должен быть погашен полностью
    \end{itemize}
    Найдите общую сумму выплат после полного погашения кредита.
\end{enumerate}
\subsection{Вклады}
    \begin{enumerate}[start=1,label={\itshape\bfseries \arabic*.}]
    \item \emph{Банковские вклады:}\\
    Владимир поместил в банк 3600 тысяч рублей под 10$\%$ годовых. В конце каждого из первых двух лет хранения после начисления процентов он дополнительно вносил на счет одну и ту же фиксированную сумму. К концу третьего года после начисления процентов оказалось, что размер вклада увеличился по сравнению с первоначальным на 48,5$\%$. Какую сумму Владимир ежегодно добавлял к вкладу?
    \item \emph{Инвестиции в бизнес:}\\
    По бизнес-плану предполагается вложить в четырёхлетний план \emph{целое} число миллионов рублей. По итогам каждого года планируется прирост средств вкладчика на 20$\%$ по сравнению с началом года. Начисленные проценты остаются вложенными в проект. Кроме этого, сразу после начислений процентов нужны дополнительные вложения: по 20 миллионов рублей в первый и второй годы, а также по 10 миллионов в третий и четвертый годы. Найдите наименьший размер первоначальных вложений, при котором общая сумма средств вкладчика за два года станет больше 125 миллионов, а за четыре года станет больше 200 миллионов рублей.
    \end{enumerate}
    \subsection{Акции}
    \begin{enumerate}[start=1,label={\itshape\bfseries \arabic*.}]
    \item \emph{Акции и ценные бумаги:}\\
    Пенсионный фонд владеет акциями на условиях, что в $n$-й год их владения сумма денег, которые стоят акции, равна $n^2$ млн. рублей. В какой-то момент деньги можно забрать и положить на вклад под 25$\%$ годовых. В какой наименьший год это выгодно сделать?
    \item \emph{Акции:}\\
    В начале года Алексей приобрёл ценные бумаги на сумму 9 тысяч рублей. В середине каждого года стоимость ценных бумаг возрастает на 2 тысячи рублей. В любой момент Алексей может продать ценные бумаги и положить вырученные им деньги на банковский счет. В середине каждого года сумма на счете будет увеличиваться на 9$\%$. В начале какого года после покупки Алексей должен продать ценные бумаги, чтобы через двадцать лет после покупки ценных бумаг сумма на банковском счете была наибольшей?
    \item \emph{Брокеры и торги на бирже:}\\
    Два брокера купили акции одного достоинства на сумму 3640 рублей. Когда цена на эти акции возросла, они продали часть акций на сумму 3927 рублей. Первый брокер продал 75$\%$ своих акций, а второй  80$\%$ своих. При этом сумма от продажи акций, полученная вторым брокером, на 140 $\%$ превысила сумму, полученную первым брокером. На сколько процентов возросла цена этой одной акции?
    \end{enumerate}
\subsection{Оптимизация в экономике}
\begin{enumerate}[start=1,label={\itshape\bfseries \arabic*.}]
    \item \emph{Оптимизация производства:}\\
    У фермера есть два поля, каждое площадью 10 гектаров. На каждом поле можно выращивать картофель и свёклу, поля можно делить между этими культурами в любой пропорции. Урожайность картофеля на первом поле составляет 300 центнеров на гектар, а на втором -- 200 центнеров на гектар. В то же время, урожайность свёклы на первом поле составляет 200 центнеров на гектар, а на втором -- 300 центнеров на гектар.  Фермер может продавать картофель по цене 10000 рублей за центнера свёклу -- по цене 13000 рублей за центнер. Какой наибольший доход может получить фермер?
    \item \emph{Разделяй и властвуй:}\\
    У Бориса есть 2 завода. Если рабочие трудятся $t^2$ часов, они производят $t$ единиц товара. За 1 час работы Борис платит рабочим \emph{первого} завода 500 рублей, а рабочим \emph{второго} завода 200 рублей. Борису необходимо, чтоб за неделю производилось 70 единиц товара. Какую наименьшую сумму денег ему придется тратить?
    \item \emph{Разделяй и властвуй:}

    В двух областях есть по 20 рабочих, каждый из которых трудится по 10 часов в сутки на добыче алюминия и никеля.  В первой области один рабочий за час добывает $0,1$ кг. алюминия или $0,1 $кг. никеля. Во второй области для добычи $x$ кг. алюминия в день требуется $x^2$ человеко-часов труда. Обе области поставляют добытый металл на завод, где для нужд промышленности производится сплав алюминия и никеля, в котором на 3 кг. алюминия приходится 1 кг. никеля. При этом области договариваются между собой вести добычу металлов так, что чтобы завод мог произвести наибольшее количество сплава. Сколько килограммов сплава при таких условиях ежегодно сможет произвести завод?
    \item \emph{Оптимальное производство:}\\
    Строительство нового завода стоило 132 миллиона рублей. Затраты на производство $x$ тысяч единиц продукции на таком заводе равны $0,5 x^2 + 5x + 17$ миллионов рублей в год. Если продукцию завода продать по цене $p$ тысяч рублей, то прибыль фирмы в миллионах рублей составит за год $px - (0,5x^2 + 5x + 17)$. Когда завод будет построен, фирма будет выпускать продукцию в таком количестве, чтобы прибыль была наибольшей. При каком наименьшем значении $p$ строительство завода окупится не более чем за 4 года?
    \end{enumerate}
    \newpage
    \subsection{Различные экономические сюжеты}
    \begin{enumerate}[start=1,label={\itshape\bfseries \arabic*.}]
    \item \emph{Уровень жизни в соседних регионах:}

    В регионе $A$ среднемесячный доход на душу населения в 2014г. составил 43470 рублей и ежегодно увеличивался на 25$\%$. В регионе $B$ среднемесячный доход на душу населения составил в 2014 году 60000 рублей. В течение трёх лет суммарный доход жителей региона $B$ увеличивался на 17 $\%$ ежегодно, а население региона увеличивалось на $m\%$ ежегодно. В 2017 году среднемесячный доход на душу населения в регионах $A$ и $B$ стал одинаковым.
    \end{enumerate}
\newpage
\section{Планиметрия}
\subsection{Повторение, основные конструкции}

\subsubsection*{Тренировочные задачи, близкие по сложности к ЕГЭ-16}
 \begin{enumerate}[start=1,label={\itshape\bfseries \arabic*.}]
	\item Катет прямоугольного треугольника равен 2, а противолежащий ему угол равен $30^{\circ}$. Найдите расстояние между центрами окружностей, вписанных в треугольники, на которые данный треугольник делит медиана, проведенная из вершины прямого угла.
	\item Диагонали трапеции взаимно перпендикулярны, а средняя линия равна 5. Найдите отрезок, соединяющий середины оснований.
	\item Расстояние между серединами взаимно перпендикулярных хорд $AC$ и $BC$ некоторой окружности равно 10. Найдите диаметр окружности.
	\item Известно, что высота трапеции равна 15, а диагонали трапеции равны 17 и 113.
	\item В равнобедренном треугольнике основание и боковая сторона равны соответственно 5 и 20. Найдите биссектрису треугольника, проведённую из вершины угла при основании.
	\item Две окружности касаются друг друга внутренним образом. Известно, что два радиуса большей окружности, угол между которыми равен $60^{\circ}$, касаются меньшей окружности. Найдите отношение радиусов окружностей.
	\item Через точку C проведены две прямые, касающиеся заданной окружности в точках A и B. На большей из дуг $AB$ взята точка $D$, для которой $CD = 2$ и $\sin{\widehat{ACD}} \cdot \sin{\widehat{BCD}} = \cfrac{1}{3}$. Найдите расстояние от точки D до хорды $AB$.
\end{enumerate}
\subsubsection*{Задачи ЕГЭ-16:}
\begin{enumerate}[start=1,label={\itshape\bfseries \arabic*.}]
 \item В остроугольном треугольнике ABC провели высоту $CC_1$ и медиану $AA_1$. Оказалось, что точки $A$, $A_1$, $C$, $C_1$ лежат на одной окружности.\\
а) Докажите, что треугольник $ABC$ равнобедренный.\\
б) Найдите площадь треугольника $ABC$, если $AA_1 : CC_1 = 5 : 4$ и $A_1C_1 = 4$.
\end{enumerate}


%%%%% КОНЕЦ ВОСЬМОГО ЛИСТОЧКА




    \subsection{Отношения, теорема Менелая и теорема Чевы}
    \subsubsection*{Дорешиваем полезную задачу}

    \begin{enumerate}[start=1,label={\itshape\bfseries \arabic*.}]

    \item На сторонах $AB$, $BC$ и $AC$ треугольника $ABC$ отмечены точки $C_1$, $A_1$ и $B_1$ соответсвенно, причем $AC_1$ : $C_1B$ = 8 : 3, $BA_1$ : $A_1 C $ = 1 : 2, $CB_1$ : $B_1 A$ = 3 : 1. Отрезки $BB_1$ и $CC_1$ пересекаются точке $D$.\\
а) Докажите, что $A D A_1 B_1$ — параллелограмм.\\
Б) Найдите $CD$, если отрезки $AD$ и $BC$ перпендикулярны, $AC = 28$, $BC = 18$.

    \end{enumerate}

    \subsubsection*{Теорема Менелая и теорема Чевы}

    \begin{enumerate}[start=1,label={\itshape\bfseries \arabic*.}]
    \item В треугольнике $ABC$ на стороне $AB$ расположена точка $K$ так, что $AK$ : $KB$ = 3 : 5. На прямой $AC$ взята точка $E$ так, что $AE$ = $2CE$. Известно, что прямые $BE$ и $CK$ пересекаются в точке $O$. Найдите площадь треугольника $ABC$, если площадь треугольника $BOC$ равна 20.
    \item В прямоугольном треугольнике $ABC$ с прямым углом $C$ точки $M$ и $N$ — середины катетов $AC$ и $BC$ соответственно, $CH$ -- высота.\\
    а) Докажите, что прямые $(MH)$ и $(NH)$ перпендикулярны.\\
    б) Пусть $P$ -- точка пересечения прямых $(AC)$ и $(NH)$, а $Q$ -- точка пересечения прямых $(BC)$ и $(MH)$. Найдите площадь треугольника $PQM$, если $AH = 4$ и $BH = 2$.

    \item В треугольнике $ABC$ из вершин к противоположным им сторонам проведены отрезки $AA_1$, $BB_1$, $CC_1$, пересекающиеся в точке $O$. $AB_1$ : $B_1C$ = $AC_1$ : $C_1B$ = 1 : 4.\\
    а) Докажите, что $AA_1$ -- медиана.\\
    б) Найдите площадь четырехугольника $AC_1OB_1$, если известно что $AC_1 = 4$, $BA_1 = 3\sqrt{2}$, $\widehat{ABC} = 45^{\circ}$.

    \item На отрезке $BD$ взята точка $C$. Биссектриса $BL$ равнобедренного треугольника $ABC$ с основанием $BC$ является боковой стороной равнобедренного треугольника $BLD$ с основанием $BD$.\\
    а) Докажите, что треугольник $DCL$ равнобедренный.\\
    б) Известно, что  косинус $\cos{\angle ABC} = \frac{1}{6}$. В каком отношении прямая $DL$ делит сторону $AB$?

    \end{enumerate}

    \subsubsection*{Отрезки и отношения}
     \begin{enumerate}[start=1,label={\itshape\bfseries \arabic*.}]
        \item В равнобедренном треугольнике $ABC (AB = BC)$ проведена биссектриса $AD$. Площади треугольников $ABD$ и $ADC$ равны соответственно $A$ и $B$.Найдите $AC$.
        \item \textit{Сдвинутая теорема Чевы:}\\
        На сторонах $AB$, $BC$, $AC$ треугольника $ABC$ взяты соответственно точки $C_1$, $A_1$, $B_1$, причем $\frac{AC_1}{C_1B} = \frac{BA_1}{A_1C} = \frac{CB_1}{B_1A} = \frac{2}{1}$. Найдите площадь треугольника, вершины которого -- попарные пересечения отрезков $AA_1$, $BB_1$, $CC_1$, если площадь треугольника $ABC$ равна 1.
     \end{enumerate}



%%%%% КОНЕЦ ВОСЬМОГО ЛИСТОЧКА



\subsection{Окружности и связанные с ними углы. Окружности и четырехугольники.}

    \begin{enumerate}[start=1,label={\itshape\bfseries \arabic*.}]
        \item Точка $O$ -- центр окружности, описанной около остроугольного треугольника $ABC$, $I$ -- центр вписанной в него окружности, $H$ -- точка пересечения высот. Известно, что $\angle BAC = \angle OBC + \angle OCB$\\
        а) Докажите, что точка $I$ лежит на окружности, описанной около треугольника $\triangle BOC$.\\
        б) Найдите угол $\angle OIH$, если $\angle ABC = 55^{\circ}$.


        \item Окружность проходит через вершины $A$, $B$ и $D$ параллелограмма $ABCD$, пересекает сторону $BC$ в точках $B$ и $E$ и пересекает сторону $CD$ в точках $K$ и $D$.\\
        а) Докажите, что $AK = AE$.\\
        б) Найдите $AD$, если известно, что $CE = 10, \ DK = 9$ и $\cos\angle BAD = 2/5$.


        \item Дана трапеция $ABCD$ с основаниями $AD$ и $BC$. Точки $M$ и $N$ -- середины сторон $AB$ и $CD$ соответственно. Окружность проходит через точки $B$ и $C$ и пересекает отрезки $BM$ и $CN$ в точках $P$ и $Q$, отличных от концов отрезка, соответственно. \\
        а) Докажите, что точки $M$, $N$, $P$ и $Q$ лежат на одной окружности.\\
        б) Найдите $PM$, если отрезки $AQ$ и $BQ$ перпендикулярны, а $AB = 15, \ BC = 1, \ CD = 17, AD = 9$.

        \item Около $\triangle ABC$ описана окружность. Прямая $BO$, где $O$ -- центр вписанной окружности, вторично пересекает описанную окружность в точке $P$.\\
        а) Докажите, что $OA = AP$.\\
        б) Найдите расстояние от точки $P$ до прямой $AC$, если $\angle ABC = 120^{\circ}$, а радиус описанной окружности равен 18.

        \item Сторона $CD$ прямоугольника $ABCD$ касается некоторой окружности в точке $M$. Продолжение стороны $AD$ пересекает окружность в точках $P$ и $Q$, причем точка $P$ лежит между точками $D$ и $Q$. Прямая $BC$ касается окружности, а точка $Q$ лежит на прямой $BM$.\\
        а) Докажите, что $\angle DMP = \angle CBM$.\\
        б) Известно, что $CM = 17$ и $CD = 25$. Найдите сторону $AD$.


        \item В трапеции $ABCD$ угол $\angle BAD$ -- прямой. Окружность, построенная на большем основании $AD$, как на диаметре, пересекает меньшее основание $BC$ в точках $C$ и $M$.\\
        а) Докажите, что $\angle BAM = \angle CAD$.\\
        б) Диагонали трапеции $ABCD$ пересекаются в точке $O$.Найдите площадь треугольника $\triangle AOB$, если $AB = 6$, а $BC = 4BM$.


        \item Дана трапеция $KLMN$ с основаниями $KN$
        и $LM$. Около треугольника $KLN$ описана окружность, прямые $LM$ и $MN$ -- касательные к этой окружности. \\
        а) Докажите, что треугольники $LMN$ и $KLN$ подобны.\\
        б) Найдите площадь треугольника $KLN$, если известно, что $KN = 3$, а $\angle LMN = 120^{\circ}$.
 \end{enumerate}



%%%%% КОНЕЦ ОЧЕРЕДНОГО ЛИСТОЧКА





\subsection{Разнобой}


    \begin{enumerate}[start=1,label={\itshape\bfseries \arabic*.}]
        \item В трапецию $ABCD$ с основаниями $AD$ и $BC$ вписана окружность с центром $O$.\\
        а) Докажите, что  $\sin\angle AOD= \sin\angle BOC$.\\
        б) Найдите площадь трапеции, если $\angle BAD = 90^{\circ}$ , а основания равны 5 и 7.
        \item Две окружности с центрами $O_1$ и $O_2$ пересекаются в точках $A$ и $B$, причём точки $O_1$ и $O_2$ лежат по разные стороны от прямой $(AB)$. Продолжения диаметра $CA$ первой окружности и хорды $CB$ этой окружности пересекают вторую окружность в точках $D$ и $E$ соответственно.\\
        а) Докажите, что треугольники $\triangle CBD$ и $\triangle O_1 A O_2$ подобны.\\
        б) Найдите $AD$, если $\angle DAE = \angle BAC$, радиус второй окружности втрое больше радиуса первой и $AB = 3$.
        \item Точка $M$ -- середина стороны $AD$ параллелограмма $ABCD$. Из вершины $A$ проведены два луча, которые разбивают отрезок $BM$ на три равные части.\\
        а) Докажите, что один из лучей содержит диагональ параллелограмма.\\
        б) Найдите площадь четырёхугольника, ограниченного двумя проведёнными лучами и прямыми $(BD)$ и $(BC)$ , если площадь параллелограмма $ABCD$ равна 40.
        \item В остроугольном треугольнике $\triangle ABC$ проведены высоты $AK$ и $CM$. На них из точек $М$ и $К$ опущены перпендикуляры $ME$ и $KH$ соответственно.\\
        а) Докажите, что прямые $(EH)$ и $(AC)$ параллельны.\\
        б) Найдите отношение $EH : AC$, если угол $\angle ABC$ равен $30^{\circ}$.\\
        \textit{Указание.} Вспомнить факты о конструкции (остроугольный треугольник и две высоты в нём), которые обсуждались осенью.
        \item Прямая, проходящая через вершину $B$ прямоугольника $ABCD$ перпендикулярно диагонали $AC$ пересекает сторону $AD$ в точке $M$, равноудалённой от вершин $B$ и $D$.\\
        а) Докажите, что $\angle ABM = \angle DBC = 30^{\circ}$.\\
        б) Найдите расстояние от центра прямоугольника до прямой $(CM)$, если $BC = 9$.

 \end{enumerate}



%%%%% КОНЕЦ ОЧЕРЕДНОГО ЛИСТОЧКА





\newpage
\section{Задачи с параметром}
\subsection{Исследование корней квадратного трехчлена}
\subsubsection*{Графический метод}
 \begin{enumerate}[start=1,label={\itshape\bfseries \arabic*.}]
    \item Найдите значения параметра $a$, при которых уравнение \begin{center} $\cfrac{9x^2 - a^2}{3x - 9 - 2a} = 0$ \end{center} имеет ровно два различных решения.
    \item Найдите значения параметра $a$, при которых уравнение \begin{center} $\log_{x^2 - 1}{(x + a)} = 1$ \end{center} имеет единственное решение.
\end{enumerate}
\subsubsection{Исследование корней квадратного трехчлена, тренировка}
 \begin{enumerate}[start=1,label={\itshape\bfseries \arabic*.}]
    \item Найдите все значения параметра $a$, при каждом из которых уравнение \begin{center}$(a - 2)x^2 +2(a - 6)x - 2a - 18 = 0 $\end{center}
    имеет два различных корня, каждый из которых больше -4.
    \item Найдите все значения параметра $a$, при каждом из которых уравнение \begin{center} $(25 - a^2)x^2 - (4a^2 - a - 7)x + a - 5=0$ \end{center} имеет два корня разных знаков.
    \item  Найдите все значения параметра $a$, при каждом из которых уравнение \begin{center}$(a - 5)x^2 + 2(a + 1)x - 2a + 1 = 0$ \end{center} имеет два различных корня, принадлежащих интервалу $(-1; 2)$.
    \item Найдите все значения параметра a, при каждом из которых уравнение \begin{center}$(a - 4)x^2 - 6(a - 2)x + 7a - 10 = 0$\end{center} имеет хотя бы один корень, меньший $3$.
    \item Найдите все значения параметра a, при каждом из которых неравенство \begin{center}$4x^2 + 2 (a - 2)x + a^2 + 2a - 8 < 0$\end{center} будет выполнено для любого значения $x$, принадлежащего интервалу $(0; 2)$.
     \item Найдите все значения параметра $a$, при каждом из которых уравнение \begin{center}
        $(ax^2 - (a^2 + 16)x + 16a)\sqrt{x + 5} = 0$
    \end{center}
    имеет ровно 2 различных корня. .
    \item Найдите все значения параметра $a$, при каждом из которых уравнение \begin{center}
        $x^2 + 2(a^2 + 7a + 3)x + 9 = 0$
    \end{center} имеет два различных положительных корня.
    \item Найдите все значения параметра $a$, при каждом из которых неравенство \begin{center} $(a - 5)x^2 - (2a - 3)x + a - 2 > 0 $\end{center} имеет решения и любое его решение принадлежит отрезку $[-\frac{1}{4}, \frac{1}{4}]$.
    \item Найдите все значения параметра $a$, при каждом из которых уравнение \begin{center} $ (a - 4)x^2 - 6(a - 2)x + 7a - 10 = 0$\end{center} имеет хотя бы один корень, меньший 3.
    \item Найдите все значения параметра $a$, при каждом из которых неравенство \begin{center}$x^2 + (a - 4)x + a^2 -2a - 8 < 0$\end{center} будет выполнено для любого значения $x$, принадлежащего интервалу $(0; 4)$.
\end{enumerate}
\subsubsection{Исследование корней квадратного трехчлена, экзаменационные}
 \begin{enumerate}[start=1,label={\itshape\bfseries \arabic*.}]
    \item Найдите значения параметра $a$, при которых уравнение
    \begin{center}
        $\log_2{(4^x - 3a)} - \log_2{(2^{x + 2} - a^2)} = 0$
    \end{center}
    имеет единственное решение.
    \item Найдите значения параметра $a$, при которых уравнение
    \begin{center}
        $\log_3{(x^2 + 2x - 3)} - \log_3{(2ax - a^2 + 2a)} = 0$
    \end{center}
    имеет единственное решение.
    \item Найдите значения параметра $a$, при которых уравнение
    \begin{center}
        $\log^2_3{x} - (a - 4)\log_3{x} - 6 = 0$
    \end{center}
    не имеет решений на промежутке $[3; 9)$
    \item Найдите все значения параметра $a$, при каждом из которых уравнение \begin{center} $2x^4 + (a - 2)x^3 + 2x^2 + (a - 2)x + 2 = 0$\end{center}
    имеет не менее двух различных отрицательных корней.
    \item Найдите все значения параметра $a$, при каждом из которых уравнение \begin{center} $\lg^2{(3x^2 + 6x + 4)} + (5a^2 - a + 4)\lg{(3x^2 + 6x + 4)} -a - 2 = 0$\end{center} имеет хотя бы один корень.
    \item Найдите все значения параметра $a$m при каждом из которых уравнение \begin{center}$36^x - 2(a + 1) \cdot 3^x + a^2 + 4a - 12 = 0$ \end{center} имеет единственный корень.
    \item Найдите все значения параметра $a$, при каждом из которых неравенство \begin{center}
        $5^x - (a - 5) \cdot (0,2)^x + 2 \le a$
    \end{center}
    имеет хотя бы одно решение.
    \item Найдите все значения параметра $a$, при каждом из которых
\end{enumerate}



%%%%% КОНЕЦ ДЕВЯТОГО ЛИСТОЧКА




\subsubsection{Метод интервалов на плоскости}
 \begin{enumerate}[start=1,label={\itshape\bfseries \arabic*.}]
    \item Надоите все значения параметра $a$, при каждом из которых множество решений неравенств
    \begin{center}
        $\begin{cases}
            a + 2x + 4 \ge x^2 \\
            a + 5 \le 4|x|
        \end{cases}$
    \end{center}
    состоит из числовой прямой и не принадлежащей ему точки этой прямой.
    \item Найдите все значения параметра $a$, при которых неравенство
    \begin{center}
        $\log_{x + 1}{(ax - x^2)} < \log_{x + 1}{x}$
    \end{center}
    верно для любого $x$ из промежутка $[-2; 0)$.
    \item Найдите все значения параметра $a$, при каждом из которых система неравенств
    \begin{center}
    $\begin{cases}
        |a - 4x + 5| \ge 18\\
        x^2 + a \le 8x + 9
    \end{cases}$
    \end{center}
    имеет единственное решение.
\end{enumerate}



%%%%% КОНЕЦ ДЕСЯТОГО ЛИСТОЧКА





\subsection{Аналитическая геометрия в задачах с параметром}

\subsubsection*{Классические задачи}
 \begin{enumerate}[start=1,label={\itshape\bfseries \arabic*.}]
    \item Найдите все значения параметра $a$, при которых система
    \begin{equation*}
    \begin{cases}
    (x - 3)^2 + (y - 6)^2 = 25\\
    y = |x - a| + 1
    \end{cases}
    \end{equation*}
    имеет ровно 3 решения.
    \item Найдите все значения параметра $a$, при которых система
    \begin{equation*}
        \begin{cases}
            y = \sqrt{16 - 6x - x^2}\\
            x = a(y - 4)
        \end{cases}
    \end{equation*}
    имеет ровно 2 решения.
    \item Найдите все значения параметра $a$, при которых система неравенств
    \begin{equation*}
        \begin{cases}
            x \le \sqrt{2x - y^2 + 2}\\
            y \ge 4 - ax
        \end{cases}
    \end{equation*}
    \item \textit{(Основная волна, 2018г.)}\\
    Найдите все значения параметра $a$, при каждом из которых система
    \begin{equation*}
        \begin{cases}
            ax^2 + ay^2 - (2a - 5)x + 2ay + 1 = 0\\
            x^2 + y = xy + x
        \end{cases}
    \end{equation*}
    имеет ровно 4 различных решения.
    \item \textit{(Основная волна, 2020г., СПБ)}\\
    Найдите все значения параметра $a$, при каждом из которых система
    \begin{equation*}
        \begin{cases}
            \log_{239}{(16 - y^2)} = \log_{239}{(16 - a^2x^2)}\\
            x^2 + y^2 = 2x + 4y
        \end{cases}
    \end{equation*}
    имеет ровно 2 различных решения.
    \item \textit{(Резервный день, 2020г.)}\\
    Найдите все значения параметра $a$, при каждом из которых система
    \begin{equation*}
        \begin{cases}
            x^2 + y^2 = 4 + 2ax - a^2\\
            x^2 = y^2
        \end{cases}
    \end{equation*}
    имеет ровно 4 различных решения.

\end{enumerate}
\subsubsection*{Круги стоят и двигаются}
 \begin{enumerate}[start=1,label={\itshape\bfseries \arabic*.}]
    \item Найдите все значения параметра $a$, при которых система
    \begin{equation*}
        \begin{cases}
            (x - 1 - 4a)^2 + (y - 1 - 3a)^2 = 9a^2 \\
            (x - 5)^2 + (y - 3)^2 = 4
        \end{cases}
    \end{equation*}
    имеет единственное решение.
    \item Найдите все значения параметра $a$, при которых система:
    \begin{equation*}
        \begin{cases}
            |3x - y + 2| \le 12\\
            (x - 3a)^2 + (y + a)^2 = 3a + 4
        \end{cases}
    \end{equation*}
    имеет единственное решение.
    \item Найдите все значения параметра $a$, при которых система
    \begin{equation*}
        \begin{cases}
            (y + 2x)(x + 2y) \le 0\\
            \sqrt{(x - a)^2 + (y - a)^2} = \cfrac{|a + 1|}{\sqrt{5}}
        \end{cases}
    \end{equation*}
    имеет ровно два решения.
\end{enumerate}



%%%%% КОНЕЦ ОДИННАДЦАТОГО ЛИСТОЧКА





\subsection{Тригонометрия в задачах с параметром}

\subsubsection*{Параметры на тригонометрическом круге}
 \begin{enumerate}[start=1,label={\itshape\bfseries \arabic*.}]
    \item (\textit{Досрочная волна, 2018г.:})\\
    Найдите все значения параметра $a$, при каждом из которых уравнение
    $$\cos{x} + 2\sin{x} = a$$
    имеет 2 решения на промежутке $[- \pi / 4, 3\pi / 4]$.
    \item Найдите все значения параметра $a$, при каждом из которых у уравнения
    $$\cfrac{2 - (4 - 4a)\sin{t}}{\cos{t} - 4\sin{t}} = 1$$
    существуют решения на промежутке $[-3\pi, -5\pi / 2]$.
\end{enumerate}
\subsubsection*{Ограниченность и квадратные уравнения}
 \begin{enumerate}[start=1,label={\itshape\bfseries \arabic*.}]
    \item Найдите все значения параметра $a$, при каждом из которых уравнение
    $$ 2\cos^2{(2^{2x - x^2 - 1})} = a + \sqrt{3}\sin{(2^{2x - x^2})}$$
    имеет решения.
    \item Найдите все значения параметра $a$, при каждом из которых область значения функции
    $$f(x) = \cfrac{\sin{x} + a}{\cos{2x} - 2}$$
    содержит число 2.
    \item Найдите все значения параметра $a$, при каждом из которых множество решений неравенства
    $$\cfrac{11a - (a^2 - 7a + 17)\sin{x} + 0}{3\cos^2{x} + a^2 + 2} < 3$$
    содержит отрезок $[0, 3\pi / 4]$.
    \item Найдите все значения параметра $a$, при каждом из которых уравнение
    $$(a + 1)\tg^2{x} - \cfrac{\tg{x}}{\cos{x}} + a = 0$$
    имеет единственное решение на $[-\frac{\pi}{6}, \frac{\pi}{2}]$.
\end{enumerate}



%%%%% КОНЕЦ ДВЕНАДЦАТОГО ЛИСТОЧКА




\subsection{Свойства функций}

\subsection*{Монотонность}
 \begin{enumerate}[start=1,label={\itshape\bfseries \arabic*.}]
    \item Найдите все значения параметра $a$, при каждом из которых уравнение
    $$27 x^6 + (4a - 2x)^3 + 6x^2 + 8a = 4x$$
    не имеет корней.
    \item Найдите все значения параметра $a$, при каждом из которых
    $$\sin{(x + 4a)} + \sin{\bigg(\cfrac{x^2 - 6x - 7a}{2}\bigg)} = 4x - x^2 - a$$
    не имеет вещественных корней.
    \item Найдите все значения параметра $a$, при каждом из которых уравнение
    $$a^2 + 11|x + 2| + 3\sqrt{x^2 + 4x + 13} = 5a + 2|x - 2a + 2|$$
    имеет хотя бы один корень.
    \item Найдите все значения параметра $b$, при каждом из которых уравнение
    $$x^3 + 2x^2 - x\log_2{(b - 1)} + 4 = 0$$
    имеет единственное решение на отрезке $[-1, 2]$.
    \item Найдите все значения параметра $a$, при которых уравнение
    $$\sin^{14}{x} + (a - 3\sin{x})^7 + \sin^2{x} + a = 3\sin{x}$$
    имеет хотя бы одно решение.
\end{enumerate}
\subsection*{Ограниченность и оценки}
 \begin{enumerate}[start=1,label={\itshape\bfseries \arabic*.}]
    \item Найдите все значения параметра $a$, при каждом из которых система
    $$\begin{cases}|y| + a = 6\sin{x} \\ y^4 + z^2 = 6a \\ (a - 3)^2 = |z^2 + 6z| + \sin^2{2x} + 9 \end{cases}$$
    имеет хотя бы одно решение, и укажите решения системы для каждого из найденных значений $a$.
    \item Найдите все значения параметра $a$, при каждом из которых число корней уравнения $|x^2 -5x + 6| = a$ равно наименьшему значению выражения $|x - a| + |2x - a| + 4|x - 1| + 1$.
    \item \textit{(Основная волна, 2013г.)}\\
    Найдите все значения параметра $a$, при каждом из которых существует хотя бы одна пара чисел $x$ и $y$, удовлетворяющих неравенству
    $$5|x - 2| + 3|x + a| \le \sqrt{4 - y^2} + 7$$
    \item Найдите все значения параметра $a$, при каждом из которых уравнение
    $$ a^2 + 10|x| + 5\sqrt{3x^2 + 25} = 5a + 3|3x - 5a|$$
    имеет хотя бы одно решение.
\end{enumerate}
\subsection*{Инвариантность}
 \begin{enumerate}[start=1,label={\itshape\bfseries \arabic*.}]
    \item Найдите все значения параметра $a$, при каждом из которых уравнение $$x^2 - 2a\sin{(\cos{x})} + a^2 = 0$$
    имеет единственный корень.
    \item Найдите все значения параметра $a$, при каждом из которых система уравнений
    $$\begin{cases} 3 \cdot 2^{|y|} + 5|y| + 3x + 4 5y^2 + 3a \\ x^2 + y^2 = 1\end{cases}$$
    имеет единственное решение.
    \item Найдите все значения параметра $a$, при которых система
    $$ \begin{cases} 2^{\ln(y)} = 4^{|x|} \\ \log_{2}(x^4 y^2 + 2a^2) = \log_2(1 - ax^2y^2) + 1\end{cases}$$
    имеет единственное решение и решите систему при этих значениях $a$.
    \item Найдите все значения параметра $a$, при каждом из которых уравнение
    $$x^2 - |x - a + 6| = |x + a - 6| - (a - 6)^2$$
    имеет единственный корень.
    \item Найдите все значения параметра $a$, при каждом из которых система уравнений
    $$\begin{cases}
    x^2 + y^2 = a^2 \\
    xy = a^2 - 3a
    \end{cases}$$
    имеет ровно 2 решения.
\end{enumerate}



%%%%%%% ЕЩЕ ЛИСТОЧЕК






\subsection{Аналитические методы в задачах с параметром}
\subsubsection{Иррациональные уравнения и неравенства}
\begin{enumerate}[start=1,label={\itshape\bfseries \arabic*.}]
    \item Найдите все значения параметра $a$, при каждом из которых уравнение
    $$\sqrt{x + a} + \sqrt{a - x} = 2a$$
    имеет хотя бы одно решение.
    \item Найдите все значения параметра $a$, при каждом из которых уравнение
    $$\sqrt{x + 2a - 1} + \sqrt{x - a} = 1$$
    имеет хотя бы одно решение.
    \item Найдите все значения параметра $a$, при каждом из которых уравнение
    $$x\sqrt{x - a} = \sqrt{4x^2 - (4a + 2)x + 2a}$$
    имеет ровно 1 корень на отрезке $[0, 1]$.
    \item Найдите все значения параметра $a$, при каждом из которых уравнение $$\sqrt{x^4 - 4x^2 + 9a^2} = x^2 + 2x - 3a$$ имеет ровно 3 решения.
    \item Найдите все значения параметра $a$, при каждом из которых уравнение
    $$|x^2 - a^2| = |x + a|\sqrt{x^2 - ax + 4a}$$
    имеет ровно два различных корня.
\end{enumerate}
\subsubsection{Нестандартные аналитические методы в исследовании квадратного трехчлена}
\begin{enumerate}[start=1,label={\itshape\bfseries \arabic*.}]
    \item Найдите все значения параметра $a$, при каждом из которых уравнение
    $$\bigg( x + \frac{1}{x + a} \bigg)^2 - (a + 9)\bigg(x + \frac{1}{x + a} \bigg) + 2a(9 - a) = 0$$
    имеет ровно 4 решения.
    \item Найдите все значения параметра $a$, при каждом из которых уравнение
    $$|x^2 - 2ax + 7| = |6a - x^2 - 2x - 1|$$
    имеет более двух корней.
\end{enumerate}
\subsubsection{Аналитическое решение систем уравнений и неравенств с параметром}
\begin{enumerate}[start=1,label={\itshape\bfseries \arabic*.}]
    \item Определите все значения параметра $a$ при каждом из которых система
    $$\begin{cases} 4^x - 2^{x + 1} = a + 3 \\ \log_2{(3 - x)} \ge a + 4\end{cases}$$
    имеет ровно 2 решения.
    \item Найдите все значения параметра $a$, при каждом из которых система уравнений
    $$\begin{cases} y = (a + 2)x^2 + 2ax + a - 2 \\ y^2  = x^2 \end{cases}$$
    имеет ровно 4 решения.
    \item Найдите все значения параметра $a$, при каждом из которых система уравнений
    $$\begin{cases} ax^2 + ay^2 - (2a - 5)x + 2ay + 1 = 0 \\ x^2 + y = xy + x\end{cases}$$
    имеет ровно 4 различных решения.
\end{enumerate}
\subsubsection{Аналитическое исследование элементарных функций}
\begin{enumerate}[start=1,label={\itshape\bfseries \arabic*.}]
    \item Найдите все значения параметра $a$, при каждом из которых уравнение
    $$x^3 + 4x^2 - ax + 6 = 0$$
    имеет единственный корень на отрезке $[-2, 2]$.
    \item Найдите все значения параметра $a$, при каждом из которых уравнение $2^x - a = \sqrt{4^x - 3a}$ имеет единственный корень.
    \item Найдите все значения параметра $a$, при каждом из которых неравенство
    $$\log_{\frac{1}{a}}(\sqrt{x^2 + ax + 5} + 1) \cdot \log_5(x^2 + ax + 6) + \log_a{3} \ge 0$$

\end{enumerate}

%%%%% КОНЕЦ ТРИНАДЦАТОГО ЛИСТОЧКА



\section{Теория чисел}
\subsection{Делимость и ее свойства. Десятичная запись числа и признаки делимости}
\subsubsection*{Подготовительные задачи на свойства делимости, близкие по сложности к №18 (а, б)}
 \begin{enumerate}[start=1,label={\itshape\bfseries \arabic*.}]
	\item Придумайте 5 различных натуральных чисел, сумма которых делится на каждое из них.
	\item Найдите все такие $a \in \mathbb{N}$, что $(a^2 + 2a - 3)$ $\vdots$ $a$.
	\item Докажите, что дробь $\cfrac{6n + 7}{10n + 12}$ несократима ни при каких натуральных n.
	\item Число умножили на сумму его цифр и получили 2008. Найдите это число.
	\item Произведение двух чисел, каждое из которых не делится нацело на 10, равно 1000. Найдите сумму этих чисел.
	\item Число умножили на сумму его цифр и получили 2008. Найдите исходное число.
 \end{enumerate}
 \subsubsection*{Подготовительные задачи на признаки делимости, близкие по сложности к №18 (а, б)}
   \begin{enumerate}[start=1,label={\itshape\bfseries \arabic*.}]
 	\item На доске написано 645*7235. Замените звездочку так, чтобы полученное число делилось на 9.
	\item К числу 15 припишите слева и справа по одной цифре так, чтоб полученное число делилось на 15.
	\item Разность двух натуральных чисел умножили на их произведение. Могло ли при это получиться число 14765817541782545?
	\item Можно ли из цифр 1, 2, 3, 4, 5, 6 составить шестизначное число, делящееся на 11?
	\item Допишите к числу 523 три цифры справа так, чтобы полученное число делилось одновременно на 7, 8 и 9.
	\item Можно ли числа от 1 до 21 разбить на несколько групп, в каждой из которых наибольшее число равно сумме всех остальных цифр.
	\item Найдите наибольшее четырехзначное число, все цифры которого различны и которое делится на 2, 5, 9 и 11.
	\item В натуральном числе a переставили цифры и получили новое натуральное число b. Известно, что $a - b = 11\ldots1$ (n единиц). Найдите наименьшее возможное значение n.
	\item Произведение натурального числа и числа, записанного теми же цифрами, но в обратном порядке, равно 2430. Чему может быть равно исходное число?
	\item (*) Числа от 1 до 37 записали в строку так, что сумма любых первых нескольких чисел делится на следующее за ними число. Какое число стоит на третьем месте, если на первом стоит число 37, а на втором 1?
 \end{enumerate}



%%%%% КОНЕЦ ЧЕТЫРНАДЦАТОГО ЛИСТОЧКА




\subsection{Деление с остатком, сравнения по модулю}

\subsubsection*{Подготовительные задачи на остатки, близкие по сложности к №18 (а, б)}
 \begin{enumerate}[start=1,label={\itshape\bfseries \arabic*.}]
	\item Найдите остаток числа $2011 \cdot 2012 + 2013^2$ при делении на 7.
	\item Найдите последнюю цифру числа $5^{239} + 9^{566} - 7{366}$
	\item Докажите, что $n^3 + 2$ не делится на 9 ни при каком натуральном n.
	\item Докажите, что число 10000\ldots0005000\ldots0001 (в каждой из двух групп -- по 2012 нулей) не является кубом натурального числа.
	\item Докажите, что $2222^{5555} + 5555^{2222}$ делится на 7.
	\item Найдите наибольшее трехзначное число, дающее при делении на 3 остаток 2, при делении на 4 остаток 3, а при делении на 5 -- остаток 4.
\end{enumerate}
\subsubsection*{Подготовительные задачи на десятичную запись числа, близкие по сложности к №19 (а, б)}
 \begin{enumerate}[start=1,label={\itshape\bfseries \arabic*.}]
	\item Натуральное число умножили на удвоенное произведение его цифр. Получилось 2016. Найдите исходное число.
	\item Между цифрами двузначного числа, делящегося на 3, вставили цифру 0, а к полученному трехзначному числу прибавили удвоенную цифру его сотен. Получилось число, в 9 раз большее первончального. Найдите исходное число.
	\item В трехзначном числе поменяли местами цифры, стоящие в разрядах единиц и сотен, после чего из исходного числа вычли полученное. Докажите, что такая разность делится на 99.
	\item Сложили шесть трехзначных чисел, полученных всевозможными перестановками трех различных цифр. Докажите, что полученная сумма делится на 37.
	\item Может ли произведение всех цифр натурального числа быть равно 2010?
	\item Найдите все четырехзначные числа $\overline{abcd}$, что $\overline{abcd} + \overline{abc} + \overline{ab} + a = 2011$.
	\item Из трех различных цифр составили все возможные двузначные числа без повторений цифр в одном числе. Сумма полученных чисел оказалась равной 528. Найдите исходные цифры.
	\item Вася не заметил знак умножения между двумя трехзначными числами и записал шестизначное число, оказавшееся в 7 раз больше их произведения. Найдите исходные числа.
\end{enumerate}
\subsubsection*{Подготовительные задачи на десятичную запись числа, близкие по сложности к №18 (б, в):}
 \begin{enumerate}[start=1,label={\itshape\bfseries \arabic*.}]
	\item Найдите все двузначные числа, которые равны сумме своей цифры десятков и квадрата цифры, стоящей в разделе единиц.
	\item Существует ли 100-значное число без нулевых цифр, которое делится на сумму своих цифр?
	\item Найдите все трехзначные числа, для которых любое число, полученное перестановкой цифр, делится на 7.
	\item Сколько существует двузначных чисел, которые ровно в 9 раз больше суммы своих цифр?
	\item В натуральном числе поменяли местами две соседние цифры и из полученного числа вычли исходное. Докажите, что полученная разность всегда делится на 9.
\end{enumerate}



%%%%% КОНЕЦ ЧЕТЫРНАДЦАТОГО ЛИСТОЧКА




\subsection{НОК и НОД, основная теорема арифметики}

\subsubsection*{Подготовительные задачи разной сложности на НОК и НОД}
 \begin{enumerate}[start=1,label={\itshape\bfseries \arabic*.}]
	\item Найдите НОД чисел 384 и 288
	\item Найдите НОД чисел 787878 и 787878787878.
	\item Найдите НОК чисел 120 и 200.
	\item Найдите НОК чисел 4 и 2011.
	\item Сколько существует пар натуральных чисел, НОК которых равен 2000?
	\item Придумайте два различных натуральных числа, произведение которых делится на их сумму.
	\item Найдите все пары натуральных чисел, сумма которых равна 288, а их НОД равен 36.
	\item Про натуральные числа x и y известно, что их НОД равен 13, а НОК равен 52. Найдите эти числа.
	\item Найдите все пары натуральных чисел, сумма которых равна 667, а частное от деления их НОК на их НОД равно 120.
\end{enumerate}
\subsubsection*{Подготовительные задачи разной сложности на делители и основную теорему арифметики:}
 \begin{enumerate}[start=1,label={\itshape\bfseries \arabic*.}]
	\item Существует ли целое число, произведение цифр которого равно 2000? 2010? 2012?
	\item Произведение двух натуральных чисел, каждое из которых не делится нацело на 10, равно 1000. Найдите их сумму.
	\item Найдите количество натуральных делителей числа: а) 10 б) 20 в) 500 г) 2000
	\item Найдите наименьшее натуральное число, половина которого -- квадрат, треть -- куб, а пятая часть -- пятая степень.
	\item Найдите все натуральные числа, которые делятся на 42 и имеют 42 различных делителя.
\end{enumerate}
%%%%%%%%%%%%%




%%%%%%%%%%%%%

\section{Теория вероятностей}
\subsection{Старые и новые сложные (?) задачи}


    \begin{enumerate}[start=1,label={\itshape\bfseries \arabic*.}]
        \item Симметричную монету бросают 10 раз. Во сколько раз вероятность события <<выпадет ровно 5 орлов>> больше вероятности события <<выпадет ровно 4 орла>>?
        \item В одном ресторане в г. Тамбове администратор предлагает гостям сыграть в «Шеш-беш»: гость бросает одновременно две игральные кости. Если он выбросит комбинацию 5 и 6 очков хотя бы один раз из двух попыток, то получит комплимент от ресторана: чашку кофе или десерт бесплатно. Какова вероятность получить комплимент? Результат округлите до сотых.
        \item  В торговом центре два одинаковых автомата продают кофе. Обслуживание автоматов происходит по вечерам после закрытия центра. Известно, что вероятность события <<К вечеру в первом автомате закончится кофе>> равна 0,25. Такая же вероятность события <<К вечеру во втором автомате закончится кофе>>. Вероятность того, что кофе к вечеру закончится в обоих автоматах, равна 0,15. Найдите вероятность того, что к вечеру кофе останется в обоих автоматах.
        \item Игральную кость бросали до тех пор, пока сумма всех выпавших очков не превысила число 3. Какова вероятность того, что для этого потребовалось два броска? Ответ округлите до сотых.
        \item Ковбой Джон попадает в муху на стене с вероятностью 0,9, если стреляет из пристрелянного револьвера. Если Джон стреляет из непристрелянного револьвера, то он попадает в муху с вероятностью 0,2. На столе лежит 10 револьверов, из них только 4 пристрелянные. Ковбой Джон видит на стене муху, наудачу хватает первый попавшийся револьвер и стреляет в муху. Найдите вероятность того, что Джон промахнётся.
        \item Телефон передаёт SMS-сообщение. В случае неудачи телефон делает следующую попытку. Вероятность того, что сообщение удастся передать без ошибок в каждой отдельной попытке, равна 0,4. Найдите вероятность того, что для передачи сообщения потребуется не больше двух попыток.
        \item При подозрении на наличие некоторого заболевания пациента отправляют на ПЦР-тест. Если заболевание действительно есть, то тест подтверждает его в 86\% случаев. Если заболевания нет, то тест выявляет отсутствие заболевания в среднем в 94\% случаев. Известно, что в среднем тест оказывается положительным у 10\% пациентов, направленных на тестирование.
        При обследовании некоторого пациента врач направил его на ПЦР-тест, который оказался положительным. Какова вероятность того, что пациент действительно имеет это заболевание?
        \item Стрелок в тире стреляет по мишени до тех пор, пока не поразит её. Известно, что он попадает в цель с вероятностью 0,2 при каждом отдельном выстреле. Какое наименьшее количество патронов нужно дать стрелку, чтобы он поразил цель с вероятностью не менее 0,6?
        \item В ящике четыре красных и два синих фломастера. Фломастеры вытаскивают по очереди в случайном порядке. Какова вероятность того, что первый раз синий фломастер появится третьим по счету?
        \item В викторине участвуют 6 команд. Все команды разной силы, и в каждой встрече выигрывает та команда, которая сильнее. В первом раунде встречаются две случайно выбранные команды. Ничья невозможна. Проигравшая команда выбывает из викторины, а победившая команда играет со следующим случайно выбранным соперником. Известно, что в первых трёх играх победила команда А. Какова вероятность того, что эта команда выиграет четвёртый раунд?
        \item Турнир по настольному теннису проводится по олимпийской системе: игроки случайным образом разбиваются на игровые пары; проигравший в каждой паре выбывает из турнира, а победитель выходит в следующий тур, где встречается со следующим противником, который определён жребием. Всего в турнире участвует 16 игроков, все они играют одинаково хорошо, поэтому в каждой встрече вероятность выигрыша и поражения у каждого игрока равна 0,5. Среди игроков два друга -- Иван и Алексей. Какова вероятность того, что этим двоим в каком-то туре придётся сыграть друг с другом?

 \end{enumerate}
 \newpage
\begin{thebibliography}{99}
\bibitem{Leibson}

К.Л. Лейбсон <<Математика. 11 класс. Сборник практических заданий.>>

\bibitem{Gordin}
Р.К. Гордин <<ЕГЭ 2021. Математика. Геометрия. Планиметрия. Задача 16. Профильный уровень.>>
\bibitem{Wolfson}
Г.И. Вольфсон <<ЕГЭ 2021 Математика. Арифметика и алгебра. Задача 19. Профильный уровень>>
\bibitem{Gushin}
Д.Д. Гущин, сайт <<Решу ЕГЭ>>
\bibitem{Fipi}
Открытый банк заданий ФИПИ.
\bibitem{Yashenko}
И.В. Ященко <<ЕГЭ-2022. Математика. Профильный уровень: типовые экзаменационные варианты: 36 вариантов.>>
\end{thebibliography}
\end{document}
